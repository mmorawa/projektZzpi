\documentclass[a4paper]{article}
\usepackage{polski}
\usepackage[utf8]{inputenc}
\usepackage{url}

\title{\bf{System elektronicznej kontroli pracowników}}
\author{{\em Mateusz Morawa (kierownik projektu)}\\
{\em Łukasz Mirek}\\
{\em Grzegorz Jasiński}\\
{\em Bartłomiej Kmak}\\
{\em Łukasz Horowski}\\
}
\date{}

\begin{document}

\begin{titlepage}
\maketitle
\thispagestyle{empty}
\bigskip
\begin{center}
Zespołowe przedsięwzięcie inżynierskie\\[2mm]

Informatyka\\[2mm]

Rok. akad. 2017/2018, sem. I\\[2mm]

Prowadzący: dr hab. Marcin Mazur
\end{center}
\end{titlepage}

\tableofcontents
\thispagestyle{empty}

\newpage

\section{Opis projektu}

\subsection{Członkowie zespołu}

\begin{enumerate}
\item Mateusz Morawa (kierownik projektu).
\item Łukasz Mirek.
\item Grzegorz Jasiński.
\item Bartłomiej Kmak.
\item Łukasz Horowski.
\end{enumerate}

\subsection{Cel projektu (produkt)}

Celem projektu jest stworzenie systemu kontroli czasu pracy pracowników. Przy pomocy urządzenia Raspberry Pi oraz modułowi RFID (czytnik oraz karta magnetyczna) pracownicy zobowiązani będą identyfikowali swojej godziny przyjścia i wyjścia z pracy poprzez przyłożenie karty do czytnika. Data, godzina oraz ID pracownika będą trafiały do bazy danych. Płaca każdego pracownika będzie wyliczana na podstawie ilości przepracowanych godzin zarejestrowanych w systemie. Właściciel za pomocą strony internetowej będzie mógł sprawdzić ilu jest  pracowników na zakładzie, czas ich pracy oraz spóźnienia.

\subsection{Potencjalny odbiorca produktu (klient)}

Potencjalnym odbiorcą oferowanego przez nas systemu jest przedsiębiorca zatrudniający grupę pracowników. Przy takiej ilości osób trudno jest zarządzać ich czasem pracy bez użycia odpowiedniego systemu.  

\subsection{Metodyka}

Projekt będzie realizowany przy użyciu (zaadaptowanej do istniejących warunków) metodyki {\em Scrum}. 

\section{Wymagania użytkownika}
Lista wymagań użytkownika w postaci ,,historyjek'' (User stories). Każda historyjka opisuje jedną cechę systemu. Struktura: As a [type of user], I want [to perform some task] so that I can [achieve some goal/benefit/value] (zob. np. \cite{us}).

\subsection{User story 1}
Jako kierownik wymagam od systemu, aby przy wejściu i wyjściu z zakładu pracownicy identyfikowali się za pomocą kart magnetycznych, po to by wiedzieć ilu pracowników jest obecnie na zakładzie.

\subsection{User story 2}
Jako księgowy chcę, aby system na podstawie przepracowanych godzin wyliczał miesięczne wynagrodzenie dla każdego pracownika,abym mógł łatwiej obliczyć należną wypłatę.

\subsection{User story 3 - opcjonalna}
Jako pracownik chcę, abym mógł w zaprojektowanym systemie na bieżąco sprawdzać ilość przepracowanych godzin po to bym mógł sprawdzić szacowaną wypłatę.

\subsection{User story 4}
Jako księgowy chcę mieć możliwość drukowania zestawienia miesięcznego wynagrodzenia brutto pracowników, żeby mieć papierową dokumentację w razie kontroli.

\subsection{User story 5}
Jako kadrowy chcę mieć możliwość dodawania i usuwania nowych pracowników do bazy danych, aby w łatwy sposób uaktualnić informacje o zatrudnionych pracownikach w firmie. 

\subsection{User story 6}
Jako pracodawca chcę, aby logowanie się na zakładowej stronie internetowej była możliwa tylko przez sieć wewnętrzną, aby zapewnić bezpieczeństwo danych w firmie.

\subsection{User story 7}
Jako pracodawca chciałbym, aby każdy korzystający z zakładowej strony internetowej miał możliwość przypomnienia hasła, żeby móc je przywrócić.

\subsection{User story 8}
Jako kierownik, chciałbym mieć dostęp do wszystkich funkcji zaimplementowanych na stronie zakładowej, ponieważ w razie nagłych wypadków chcę mieć kontrolę nad całym systemem.


\section{Harmonogram}

\subsection{Rejestr zadań (Product Backlog)}

\begin{itemize}
\item Data rozpoczęcia: 25.10.17 r.
\item Data zakończenia: 08.11.17 r.
\end{itemize}

\subsection{Sprint 1}

\begin{itemize}
\item Data rozpoczęcia: 08.11.17 r.
\item Data zakończenia: 29.11.17 r.
\item Scrum Master: Łukasz Mirek.
\item Product Owner: Grzegorz Jasiński.
\item Development Team: Mateusz Morawa, Łukasz Horowski, Grzegorz Jasiński, Łukasz Mirek, Bartłomiej Kmak.
\end{itemize}

\subsection{Sprint 2}

\begin{itemize}
\item Data rozpoczęcia: 29.11.17 r.
\item Data zakończenia: 20.12.17 r.
\item Scrum Master: Łukasz Horowski.
\item Product Owner: Mateusz Morawa.
\item Development Team: Grzegorz Jasiński, Łukasz Mirek, Bartłomiej Kmak, Łukasz Horowski.
\end{itemize}

\subsection{Sprint 3}

\begin{itemize}
\item Data rozpoczęcia: 20.12.17 r.
\item Data zakończenia: 10.01.18 r.
\item Scrum Master: Mateusz Morawa.
\item Product Owner: Kmak Bartłomiej.
\item Development Team: Mateusz Morawa, Grzegorz Jasiński, Łukasz Mirek, Bartłomiej Kmak, Łukasz Horowski.
\end{itemize}

\subsection{Sprint 4}

\begin{itemize}
\item Data rozpoczęcia: 10.01.18 r.
\item Data zakończenia: 24.01.18 r.
\item Scrum Master: Bartłomiej Kmak.
\item Product Owner: Łukasz Mirek.
\item Development Team: Mateusz Morawa, Grzegorz Jasiński, Łukasz Mirek, Łukasz Horowski.
\end{itemize}

\section{Product Backlog}

\subsection{Backlog Item 1}
\paragraph{Tytuł zadania.} Rejestrowanie wejścia i wyjścia pracownika z zakładu.
\paragraph{Opis zadania.} Do realizacji powyższego zadania zostanie użyty Raspberry Pi i podłączony do niego czytnik kart magnetycznych, do którego swoje karty przykładać będą pracownicy w celu rejestracji przybycia lub opuszczenia zakładu. Następnie specjalny program napisany w języku Python będzie zgromadzone dane zapisywał do pliku i okresowo wysyłał do bazy danych na firmowym serwerze.
\paragraph{Priorytet.} 1.
\paragraph{Definition of Done.} Dane odnośnie wejścia i wyjścia odczytane są z karty pracownika przez czytnik RFID podłączony do Raspberry Pi i wysyłane do bazy danych, znajdującej się na tym urządzeniu.

\subsection{Backlog Item 2}
\paragraph{Tytuł zadania.} Obliczanie wynagrodzenia.
\paragraph{Opis zadania.} System na podstawie danych zgromadzonych w bazie, będzie zliczał godziny przepracowane przez poszczególnego pracownika, po czym wyznaczy miesięczne wynagrodzenie zgodnie z określoną stawką godzinową zawartą w umowie. Pracownicy otrzymywać będą wynagrodzenie tylko za pełne godziny przepracowane na zakładzie w przedziale czasowym od ósmej do szesnastej. Gdy pracownik przyjdzie do zakładu w trakcie trwającej godziny pracy będzie mógł otrzymać zapłatę dopiero za następną pełną godzinę.
\paragraph{Priorytet.} 2.
\paragraph{Definition of Done.} System będzie wyliczał wynagrodzenie miesięczne pracowników na podstawie przepracowanych pełnych godzin. Zliczanie rozpoczyna się od godz. 8.00, kończy o godz. 16.00. Tylko przepracowanie pełnej godziny gwarantuje otrzymanie za nią wynagrodzenia. Istotna jest punktualność.


\subsection{Backlog Item 3}
\paragraph{Tytuł zadania.} Funkcja umożliwiająca pracownikowi sprawdzenie wynagrodzenia (opcjonalna).
\paragraph{Opis zadania.} System po zalogowaniu się pracownika na stronę umożliwi mu wgląd do zestawienia z poprzednich miesięcy oraz z poszczególnych dni obecnego miesiąca. Zestawienie wygeneruje pracownikowi pensję jaką otrzymał w poprzednich miesiącach a także wyświetli szczegółowe dane co do przepracowanych godzin.
\paragraph{Priorytet.} 4.
\paragraph{Definition of Done.} Pracownik po zalogowaniu się na stronie może zobaczyć zestawienie swoich zarobków. Zestawienie będzie wyświetlane w formie tabeli. Istnieje możliwość wyboru poszczególnych miesięcy, które interesują pracownika (maksymalnie do 12 miesięcy wstecz).

\subsection{Backlog Item 4}
\paragraph{Tytuł zadania.} Drukowanie zestawienia miesięcznego wynagrodzenia brutto pracowników.
\paragraph{Opis zadania.} W sytuacji, gdy pod koniec miesiąca dział księgowy będzie musiał wydrukować dokumenty zawierające dane o miesięcznych wynagrodzeniach brutto pracowników, system po zalogowaniu się księgowego na stronie udostępni mu odpowiednie opcje do wykonania tej czynności.
\paragraph{Priorytet.} 4.
\paragraph{Definition of Done.} System umożliwia księgowym drukowanie zestawienia miesięcznego wynagrodzenia brutto pracowników.


\subsection{Backlog Item 5}
\paragraph{Tytuł zadania.} Dodawanie oraz usuwanie nowych pracowników.
\paragraph{Opis zadania.} Za pośrednictwem strony zakładowej, kadrowy będzie miał możliwość dodawania nowych pracowników oraz ich usuwania w przypadku zwolnienia. Nowi pracownicy będą dodawani do bazy danych oraz będą mieli możliwość logowania się na stronę zakładu.
\paragraph{Priorytet.} 2.
\paragraph{Definition of Done.} Kadrowy będzie mógł dodawać i usuwać pracowników z systemu za pomocą strony zakładowej.

\subsection{Backlog Item 6}
\paragraph{Tytuł zadania.} Utworzenie lokalnego serwera dla systemu.
\paragraph{Opis zadania.} Dostęp do serwera powinien być ograniczony do sieci wewnętrznej w celu zabezpieczenia przed zewnętrznymi atakami. Na serwerze znajdować się będzie usługa WWW z obsługą PHP oraz baza danych.
\paragraph{Priorytet.} 1.
\paragraph{Definition of Done.} Istnieje w pełni działający lokalny serwer, który posiada bazę danych MySQL zabezpieczoną przed nieuprawnionym dostępem za pomocą hasła dla administratora. Na serwerze zostanie zainstalowany interpreter PHP w celu uruchomienia strony internetowej zakładu, który pozwala na kontakt z bazą danych.

\subsection{Backlog Item 7}
\paragraph{Tytuł zadania.} Przywracanie hasła.
\paragraph{Opis zadania.} Strona zakładowa będzie posiadać funkcję, dzięki której pracownik będzie mógł wysłać prośbę o przypomnienie hasła w tym celu dodatkowo uruchomiona zostanie usługa pocztowa SMTP na serwerze. Wiadomość z nowym hasłem zostanie wysłana na e-mail pracownika. Po zalogowaniu się za pomocą wygenerowanego hasła, system wymusi na pracowniku ustawienie nowego hasła.
\paragraph{Priorytet.} 5.
\paragraph{Definition of Done.} Działa serwer SMTP wysyłający maile, który umożliwia pracownikom przypomienie hasła oraz jego zmianę w systemie.

\subsection{Backlog Item 8}
\paragraph{Tytuł zadania.} Uprawnienia kierownika do wszystkich funkcji.
\paragraph{Opis zadania.} W sytuacjach wyjątkowych kierownik zakładu powinien mieć możliwość dokonywania wszelkich zmian dostępnych dla innych osób poprzez stronę internetową. Odbywać się to będzie przez odpowiednią zakładkę na stronie, która będzie rejestrować każdą zmianę wraz z powodem jej wykonania.
\paragraph{Priorytet.} 3.
\paragraph{Definition of Done.} Kierownik jest w stanie dokonać następujących zmian w systemie poprzez stronę internetową:
\begin{itemize}
\item Dodawanie nowego pracownika.
\item Usuwanie istniejącego pracownika.
\item Edycja danych osobowych pracownika.
\item Możliwość sprawdzenia miesięcznego zestawienia wypłaty pracowników.
\item Zmiany stawki wynagrodzenia pracownika.
\end {itemize}

\section{Sprint 1}
\subsection{Cel} Celem pierwszego sprintu jest stworzenie początkowej wersji systemu pozwalającej rejestrować wejścia i wyjścia pracowników przy pomocy urządzenia Raspberry Pi wraz z dołączonym modułem RFID(karta magnetyczna, czytnik kart). Te dane będą przesyłane na wcześniej utworzony i skonfigurowany serwer bazy danych. Utworzona zostanie również pierwotna wersja strony, na której wyświetlane będą informacje( imię i nazwisko pracownika, data, godzina wejścia, godzina wyjścia, ID karty).
\subsection{Sprint Planning/Backlog}

\paragraph{Tytuł zadania.} Rejestrowanie wejścia i wyjścia pracownika z zakładu.
\begin{itemize}
\item Estymata: Large przy użyciu skali rozmiarów T-shirtów (13 osobogodzin).
\end{itemize}

\paragraph{Tytuł zadania.} Utworzenie lokalnego serwera dla systemu.
\begin{itemize}
\item Estymata: Large przy użyciu skali rozmiarów T-shirtów (10 osobogodzin).
\end{itemize}

\subsection{Realizacja}

\paragraph{Tytuł zadania.} Rejestrowanie wejścia i wyjścia pracownika z zakładu.
\subparagraph{Wykonawca.} Mateusz Morawa, Łukasz Horowski.
\subparagraph{Realizacja.} <<Sprawozdanie z realizacji zadania (w tym ocena zgodności z estymatą). Kod programu (środowisko \texttt{verbatim}): \begin{verbatim}
for (i=1; i<10; i++)
...
\end{verbatim}>>.

\paragraph{Tytuł zadania.} Utworzenie lokalnego serwera dla systemu.
\subparagraph{Wykonawca.} Łukasz Mirek, Łukasz Horowski, Bartłomiej Kmak, Grzegorz Jasiński.
\subparagraph{Realizacja.} <<Sprawozdanie z realizacji zadania (w tym ocena zgodności z estymatą). Kod programu (środowisko \texttt{verbatim}): \begin{verbatim}
for (i=1; i<10; i++)
...
\end{verbatim}>>.


\subsection{Sprint Review/Demo}
<<Sprawozdanie z przeglądu Sprint'u -- czy założony cel (przyrost) został osiągnięty oraz czy wszystkie zaplanowane Backlog Item'y zostały zrealizowane? Demonstracja przyrostu produktu>>.

\section{Sprint 2}

\subsection{Cel} <<Określić, w jakim celu tworzony jest przyrost produktu>>.

\subsection{Sprint Planning/Backlog}

\paragraph{Tytuł zadania.} <<Tytuł>>.
\begin{itemize}
\item Estymata: <<szacowana czasochłonność (w ,,koszulkach'')>>.
\end{itemize}

\paragraph{Tytuł zadania.} <<Tytuł>>.
\begin{itemize}
\item Estymata: <<szacowana czasochłonność (w ,,koszulkach'')>>.
\end{itemize}

\paragraph{<<Tutaj dodawać kolejne zadania>>}

\subsection{Realizacja}

\paragraph{Tytuł zadania.} <<Tytuł>>.
\subparagraph{Wykonawca.} <<Wykonawca>>.
\subparagraph{Realizacja.} <<Sprawozdanie z realizacji zadania (w tym ocena zgodności z estymatą). Kod programu (środowisko \texttt{verbatim}): \begin{verbatim}
for (i=1; i<10; i++)
...
\end{verbatim}>>.

\paragraph{Tytuł zadania.} <<Tytuł>>.
\subparagraph{Wykonawca.} <<Wykonawca>>.
\subparagraph{Realizacja.} <<Sprawozdanie z realizacji zadania (w tym ocena zgodności z estymatą). Kod programu (środowisko \texttt{verbatim}): \begin{verbatim}
for (i=1; i<10; i++)
...
\end{verbatim}>>.

\paragraph{<<Tutaj dodawać kolejne zadania>>}


\subsection{Sprint Review/Demo}
<<Sprawozdanie z przeglądu Sprint'u -- czy założony cel (przyrost) został osiągnięty oraz czy wszystkie zaplanowane Backlog Item'y zostały zrealizowane? Demostracja przyrostu produktu>>.

\section*{<<Tutaj dodawać kolejne Sprint'y>>}


\begin{thebibliography}{9}

\bibitem{Cov} S. R. Covey, {\em 7 nawyków skutecznego działania}, Rebis, Poznań, 2007.

\bibitem{Oet} Tobias Oetiker i wsp., Nie za krótkie wprowadzenie do systemu \LaTeX  \ $2_\varepsilon$, \url{ftp://ftp.gust.org.pl/TeX/info/lshort/polish/lshort2e.pdf}

\bibitem{SchSut} K. Schwaber, J. Sutherland, {\em Scrum Guide}, \url{http://www.scrumguides.org/}, 2016.

\bibitem{apr} \url{https://agilepainrelief.com/notesfromatooluser/tag/scrum-by-example}

\bibitem{us} \url{https://www.tutorialspoint.com/scrum/scrum_user_stories.htm}

\end{thebibliography}

\end{document}

% ----------------------------------------------------------------
