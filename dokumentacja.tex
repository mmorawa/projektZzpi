\documentclass[a4paper]{article}
\usepackage{polski}
\usepackage[utf8]{inputenc}
\usepackage{url}

\title{\bf{System elektronicznej kontroli pracowników}}
\author{{\em Mateusz Morawa (kierownik projektu)}\\
{\em Łukasz Mirek}\\
{\em Grzegorz Jasiński}\\
{\em Bartłomiej Kmak}\\
{\em Łukasz Horowski}\\
}
\date{}

\begin{document}

\begin{titlepage}
\maketitle
\thispagestyle{empty}
\bigskip
\begin{center}
Zespołowe przedsięwzięcie inżynierskie\\[2mm]

Informatyka\\[2mm]

Rok. akad. 2017/2018, sem. I\\[2mm]

Prowadzący: dr hab. Marcin Mazur
\end{center}
\end{titlepage}

\tableofcontents
\thispagestyle{empty}

\newpage

\section{Opis projektu}

\subsection{Członkowie zespołu}

\begin{enumerate}
\item Mateusz Morawa (kierownik projektu).
\item Łukasz Mirek.
\item Grzegorz Jasiński.
\item Bartłomiej Kmak.
\item Łukasz Horowski.
\end{enumerate}

\subsection{Cel projektu (produkt)}

Celem projektu jest stworzenie systemu kontroli czasu pracy pracowników. Przy pomocy urządzenia raspberry oraz modułowi RFID (czytnik oraz karta magnetyczna) pracownicy zobowiązani będą identyfikowali swojej godziny przyjścia i wyjścia z pracy poprzez przyłożenie karty do czytnika. Data, godzina oraz ID pracownika będą trafiały do bazy danych. Płaca każdego pracownika będzie wyliczana na podstawie ilosci przepracowanych godzin zarejestrowanych w systemie. Właściciel za pomocą strony internetowej będzie mógł sprawdzić ilu jest  pracowników na zakładzie, czas ich pracy oraz spóźnienia.

\subsection{Potencjalny odbiorca produktu (klient)}

Potencjalnym odbiorcą oferowanego przez nas systemu jest przedsiębiorca zatrudniający grupę pracowników. Przy takiej ilości osób trudno jest zarządzać ich czasem pracy bez użycia odpowiedniego systemu.  

\subsection{Metodyka}

Projekt będzie realizowany przy użyciu (zaadaptowanej do istniejących warunków) metodyki {\em Scrum}. 

\section{Wymagania użytkownika}
Lista wymagań użytkownika w postaci ,,historyjek'' (User stories). Każda historyjka opisuje jedną cechę systemu. Struktura: As a [type of user], I want [to perform some task] so that I can [achieve some goal/benefit/value] (zob. np. \cite{us}).

\subsection{User story 1}
Jako kierownik wymagam od systemu, aby przy wejściu i wyjściu z zakładu pracownicy identyfikowali się za pomocą kart magnetycznych, po to by wiedzieć ilu pracowników jest obecnie na zakładzie.

\subsection{User story 2}
Jako księgowy chcę, aby system na podstawie przepracowanych godzin wyliczał miesięczne wynagrodzenie dla każdego pracownika,abym mógł łatwiej obliczyć należną wypłatę.

\subsection{User story 3}
Jako pracownik chcę, abym mógł w zaprojektowanym systemie na bieżąco sprawdzać ilość przepracowanych godzin po to bym mógł sprawdzić szacowaną wypłatę.

\subsection{User story 4}
Jako księgowy chcę mieć możliwość drukowania zestawienia miesięcznego wynagrodzenia brutto pracowników, żeby mieć papierową dokumentację w razie kontroli.

\subsection{User story 5}
Jako kadrowy chcę mieć możliwość dodawania i usuwania nowych pracowników do bazy danych, aby w łatwy sposób uaktualnić informacje o zatrudnionych pracownikach w firmie. 

\subsection{User story 6}
Jako pracodawca chcę, aby logowanie się na zakładowej stronie internetowej była możliwa tylko przez sieć wewnętrzną, aby zapewnić bezpieczeństwo danych w firmie.

\subsection{User story 7}
Jako pracodawca chciałbym, aby każdy korzystający z zakładowej strony internetowej miał możliwość przypomnienia hasła, żeby móc je przywrócić.

\subsection{User story 8}
Jako kierownik, chciałbym mieć dostęp do wszystkich funkcji zaimplementowanych na stronie zakładowej, ponieważ w razie nagłych wypadków chcę mieć kontrolę nad całym systemem.


\section{Harmonogram}

\subsection{Rejestr zadań (Product Backlog)}

\begin{itemize}
\item Data rozpoczęcia: 25.10.17 r.
\item Data zakończenia: 08.11.17 r.
\end{itemize}

\subsection{Sprint 1}

\begin{itemize}
\item Data rozpoczęcia: 08.11.17 r.
\item Data zakończenia: 29.11.17 r.
\item Scrum Master: Mirek Łukasz.
\item Product Owner: Jasiński Grzegorz.
\item Development Team: Morawa Mateusz, Horowski Łukasz, Jasiński Grzegorz, Mirek Łukasz, Kmak Bartłomiej.
\end{itemize}

\subsection{Sprint 2}

\begin{itemize}
\item Data rozpoczęcia: 29.11.17 r.
\item Data zakończenia: 20.12.17 r.
\item Scrum Master: Horowski Łukasz.
\item Product Owner: Mateusz Morawa.
\item Development Team: Jasiński Grzegorz, Mirek Łukasz, Kmak Bartłomiej, Horowski Łukasz.
\end{itemize}

\subsection{Sprint 3}

\begin{itemize}
\item Data rozpoczęcia: 20.12.17 r.
\item Data zakończenia: 10.01.18 r.
\item Scrum Master: Morawa Mateusz.
\item Product Owner: Kmak Bartłomiej.
\item Development Team: Morawa Mateusz, Jasiński Grzegorz, Mirek Łukasz, Kmak Bartłomiej, Horowski Łukasz.
\end{itemize}

\subsection{Sprint 4}

\begin{itemize}
\item Data rozpoczęcia: 10.01.18 r.
\item Data zakończenia: 24.01.18 r.
\item Scrum Master: Jasiński Grzegorz.
\item Product Owner: Mirek Łukasz.
\item Development Team: Morawa Mateusz, Jasiński Grzegorz, Mirek Łukasz, Horowski Łukasz.
\end{itemize}

\section{Product Backlog}

\subsection{Backlog Item 1}
\paragraph{Tytuł zadania.} Rejestrowanie wejścia i wyjścia pracownika z zakładu.
\paragraph{Opis zadania.} Do realizacji powyższego zadania zostanie użyty Raspberry Pi i podłączony do niego czytnik kart magnetycznych, do którego swoje karty przykładać będą pracownicy w celu rejestracji przybycia lub opuszczenia zakładu. Następnie specjalny program napisany w języku Python będzie zgromadzone dane zapisywał do pliku i okresowo wysyłał do bazy danych na firmowym serwerze.
\paragraph{Priorytet.} 1.
\paragraph{Definition of Done.} System gromadzi dane odnośnie wejścia i wyjścia odczytane z karty  pracownika do bazy danych.

\subsection{Backlog Item 2}
\paragraph{Tytuł zadania.} Obliczanie wynagrodzenia.
\paragraph{Opis zadania.} System na podstawie danych zgromadzonych w bazie, będzie zliczał godziny przepracowane przez poszczególnego pracownika, po czym wyznaczy miesiączne wynagrodzenie zgodnie z określoną stawką godzinową zawartą w umowie. Pracownicy otrzymywać będą wynagrodzenie tylko za pełne godziny przepracowane na zakładzie w przedziale czasowym od ósmej do szesnastej. Gdy pracownik przyjdzie do zakładu w trakcie trwającej godziny pracy będzie mógł otrzymać zapłatę dopiero za następną pełną godzinę.
\paragraph{Priorytet.} 2.
\paragraph{Definition of Done.} System będzie wyliczał wynagrodzenie miesięczne pracowników.

\subsection{Backlog Item 3}
\paragraph{Tytuł zadania.} Funkcja sprawdzająca wynagrodzenie pracownika (opcjonalna).
\paragraph{Opis zadania.} System po zalogowaniu się pracownika na stronę umozliwi mu wgląd do zestawienia z poprzednich miesięcy oraz z poszczególnych dni obecnego miesiąca. Zestawienie wygeneruje pracownikowi pensję jaką otrzymał w poprzednich miesącach a także wyświetli szczegółowe dane co do przepracowanych godzin.
\paragraph{Priorytet.} 4.
\paragraph{Definition of Done.} Pracownik po zalogowaniu się na stronie może zobaczyć zestawienie swoich zarobków.

\subsection{Backlog Item 4}
\paragraph{Tytuł zadania.} Drukowanie zestawienia miesięcznego wynagrodzenia brutto pracowników.
\paragraph{Opis zadania.} W sytuacji, gdy pod koniec miesiąca dział księgowy będzie musiał wydrukować dokumenty zawierające dane o miesięcznych wynagrodzeniach brutto pracowników, system po zalogowaniu się księgowego na stronie udostępni mu odpowiednie opcje do wykonania tej czynności.
\paragraph{Priorytet.} 4.
\paragraph{Definition of Done.} System umożliwia księgowym drukowanie zestawienia miesięcznego wynagrodzenia brutto pracowników.


\subsection{Backlog Item 5}
\paragraph{Tytuł zadania.} Dodawanie oraz usuwanie nowych pracowników.
\paragraph{Opis zadania.} Za pośrednictwem strony zakładowej, kierownik będzie miał możliwość dodawania nowych pracowników oraz ich usuwania w przypadku zwolnienia. Nowi pracownicy bedą dodawani do bazy danych oraz będą mieli możliwość logowania się na stronę zakładu.
\paragraph{Priorytet.} 2.
\paragraph{Definition of Done.} Kierownik będzie mógł dodawać i usuwać pracowników z systemu za pomocą strony.

\subsection{Backlog Item 6}
\paragraph{Tytuł zadania.} Utworzenie lokalnego serwera dla systemu.
\paragraph{Opis zadania.} Dostęp do serwera powinien być ograniczony do sieci wewnętrznej w celu zabezpieczenia przed zewnętrznymi atakami. Na serwerze znajdować się będzie usługa WWW z obsługą PHP oraz baza danych. Dodatkowo uruchomiona zostanie usługa pocztowa SMTP.
\paragraph{Priorytet.} 1.
\paragraph{Definition of Done.} Istnieje w pełni działający lokalny serwer, który jest odpowiednio skonfigurowany i zabezpieczony.

\subsection{Backlog Item 7}
\paragraph{Tytuł zadania.} Przywracanie hasła.
\paragraph{Opis zadania.} Strona zakładowa będzie posiadać funkcję, dzięki której pracownik będzie mógł wysłac prośbę o przypomnienie hasła . Wiadomość z nowym hasłem zostanie wysłana na e-mail pracownika. Po zalogowaniu się za pomoca wygenerowanego hasła, system wymusi na pracowniku ustawienie nowego hasła.
\paragraph{Priorytet.} 5.
\paragraph{Definition of Done.} Pracownik ma mieć możliwość przypomienia hasła oraz jego zmiany w systemie.

\subsection{Backlog Item 8}
\paragraph{Tytuł zadania.} Uprawnienia kierownika do wszystkich funkcji.
\paragraph{Opis zadania.} W sytuacjach wyjątkowych kierownik zakładu powinien mieć możliwość dokonywania wszelkich zmian dostępnych dla innych osób poprzez stronę internetową. Odbywać się to będzie przez odpowiednią zakładkę na stronie, która będzie rejestrować każdą zmianę wraz z powodem jej wykonania.
\paragraph{Priorytet.} 3.
\paragraph{Definition of Done.} Kierownik jest w stanie dokonać odpowiednich zmian w systemie poprzez stronę internetową.

\section{Sprint 1}
\subsection{Cel} <<Określić, w jakim celu tworzony jest przyrost produktu>>.
\subsection{Sprint Planning/Backlog}

\paragraph{Tytuł zadania.} <<Tytuł>>.
\begin{itemize}
\item Estymata: <<szacowana czasochłonność (w ,,koszulkach'')>>.
\end{itemize}

\paragraph{Tytuł zadania.} <<Tytuł>>.
\begin{itemize}
\item Estymata: <<szacowana czasochłonność (w ,,koszulkach'')>>.
\end{itemize}

\paragraph{<<Tutaj dodawać kolejne zadania>>}

\subsection{Realizacja}

\paragraph{Tytuł zadania.} <<Tytuł>>.
\subparagraph{Wykonawca.} <<Wykonawca>>.
\subparagraph{Realizacja.} <<Sprawozdanie z realizacji zadania (w tym ocena zgodności z estymatą). Kod programu (środowisko \texttt{verbatim}): \begin{verbatim}
for (i=1; i<10; i++)
...
\end{verbatim}>>.

\paragraph{Tytuł zadania.} <<Tytuł>>.
\subparagraph{Wykonawca.} <<Wykonawca>>.
\subparagraph{Realizacja.} <<Sprawozdanie z realizacji zadania (w tym ocena zgodności z estymatą). Kod programu (środowisko \texttt{verbatim}): \begin{verbatim}
for (i=1; i<10; i++)
...
\end{verbatim}>>.

\paragraph{<<Tutaj dodawać kolejne zadania>>}


\subsection{Sprint Review/Demo}
<<Sprawozdanie z przeglądu Sprint'u -- czy założony cel (przyrost) został osiągnięty oraz czy wszystkie zaplanowane Backlog Item'y zostały zrealizowane? Demostracja przyrostu produktu>>.

\section{Sprint 2}

\subsection{Cel} <<Określić, w jakim celu tworzony jest przyrost produktu>>.

\subsection{Sprint Planning/Backlog}

\paragraph{Tytuł zadania.} <<Tytuł>>.
\begin{itemize}
\item Estymata: <<szacowana czasochłonność (w ,,koszulkach'')>>.
\end{itemize}

\paragraph{Tytuł zadania.} <<Tytuł>>.
\begin{itemize}
\item Estymata: <<szacowana czasochłonność (w ,,koszulkach'')>>.
\end{itemize}

\paragraph{<<Tutaj dodawać kolejne zadania>>}

\subsection{Realizacja}

\paragraph{Tytuł zadania.} <<Tytuł>>.
\subparagraph{Wykonawca.} <<Wykonawca>>.
\subparagraph{Realizacja.} <<Sprawozdanie z realizacji zadania (w tym ocena zgodności z estymatą). Kod programu (środowisko \texttt{verbatim}): \begin{verbatim}
for (i=1; i<10; i++)
...
\end{verbatim}>>.

\paragraph{Tytuł zadania.} <<Tytuł>>.
\subparagraph{Wykonawca.} <<Wykonawca>>.
\subparagraph{Realizacja.} <<Sprawozdanie z realizacji zadania (w tym ocena zgodności z estymatą). Kod programu (środowisko \texttt{verbatim}): \begin{verbatim}
for (i=1; i<10; i++)
...
\end{verbatim}>>.

\paragraph{<<Tutaj dodawać kolejne zadania>>}


\subsection{Sprint Review/Demo}
<<Sprawozdanie z przeglądu Sprint'u -- czy założony cel (przyrost) został osiągnięty oraz czy wszystkie zaplanowane Backlog Item'y zostały zrealizowane? Demostracja przyrostu produktu>>.

\section*{<<Tutaj dodawać kolejne Sprint'y>>}


\begin{thebibliography}{9}

\bibitem{Cov} S. R. Covey, {\em 7 nawyków skutecznego działania}, Rebis, Poznań, 2007.

\bibitem{Oet} Tobias Oetiker i wsp., Nie za krótkie wprowadzenie do systemu \LaTeX  \ $2_\varepsilon$, \url{ftp://ftp.gust.org.pl/TeX/info/lshort/polish/lshort2e.pdf}

\bibitem{SchSut} K. Schwaber, J. Sutherland, {\em Scrum Guide}, \url{http://www.scrumguides.org/}, 2016.

\bibitem{apr} \url{https://agilepainrelief.com/notesfromatooluser/tag/scrum-by-example}

\bibitem{us} \url{https://www.tutorialspoint.com/scrum/scrum_user_stories.htm}

\end{thebibliography}

\end{document}

% ----------------------------------------------------------------
